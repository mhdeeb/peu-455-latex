\documentclass[12pt]{article}
\usepackage[margin=1.25in]{geometry}
\usepackage[svgnames,x11names,table]{xcolor}
\usepackage{hyperref}
\usepackage{graphicx}
\usepackage{parskip}
\usepackage{float}
\usepackage{amsmath}
\usepackage{amssymb}
\usepackage{enumitem}
\usepackage[thicklines]{cancel}

\hypersetup{
    colorlinks,
    citecolor=black,
    filecolor=black,
    linkcolor=RoyalBlue4,
    urlcolor=RoyalBlue4,
}

\title{PEU 455 Assignment 3}
\author{Mohamed Hussien El-Deeb (201900052)}
\date{}

\begin{document}

\maketitle
\tableofcontents

\newcommand{\Lagr}{\mathcal{L}}
\renewcommand{\labelenumi}{\textbf{(\alph{enumi})}}
\newcommand{\mbeq}{\overset{!}{=}}

\newpage
\section{11.2.3}

\[
    \frac{\partial u}{\partial x} = \frac{\partial v}{\partial y},
    \quad \frac{\partial u}{\partial y} = -\frac{\partial v}{\partial x}
\]

\begin{enumerate}
    \item \[
              u(x, y) = x^3 - 3xy^2
          \]
          \[
              \frac{\partial u}{\partial x} = 3x^2-3y^2 = \frac{\partial v}{\partial y}
          \]
          \[
              v = \int 3x^2-3y^2\, dy = 3yx^2-y^3 + f(x) \tag{1}\label{eq.1}
          \]
          \[
              \frac{\partial u}{\partial y} = - 6xy = -\frac{\partial v}{\partial x}
          \]
          \[
              v = 3x^2y + g(y) \tag{2}\label{eq.2}
          \]
          from eq.\ref{eq.1} and eq.\ref{eq.2}
          \[
              v = 3x^2y - y^3
          \]
    \item \[
              v(x, y) = e^{-y}\sin x
          \]
          \[
              \frac{\partial v}{\partial y} = -e^{-y}\sin x = \frac{\partial u}{\partial x}
          \]
          \[
              u = e^{-y}\cos x + f(y) \tag{3}\label{eq.3}
          \]
          \[
              \frac{\partial v}{\partial x} = e^{-y}\cos x = - \frac{\partial u}{\partial y}
          \]
          \[
              u = e^{-y}\cos x + g(x) \tag{4}\label{eq.4}
          \]
          from eq.\ref{eq.3} and eq.\ref{eq.4}
          \[
              u = e^{-y}\cos x
          \]
\end{enumerate}

\newpage
\section{11.2.4}

From the fact $w$ is analytical,

\[
    \frac{\partial u}{\partial x} = \frac{\partial v}{\partial y},
    \quad \frac{\partial u}{\partial y} = -\frac{\partial v}{\partial x}
\]

From the fact $w^*$ is analytical,

\[
    \frac{\partial u}{\partial x} = -\frac{\partial v}{\partial y},
    \quad \frac{\partial u}{\partial y} = \frac{\partial v}{\partial x}
\]

We can see that,

\[
    \frac{\partial u}{\partial x} = - \frac{\partial u}{\partial x} = 0,\quad
    \frac{\partial u}{\partial y} = - \frac{\partial u}{\partial y} = 0,\quad
    \frac{\partial v}{\partial x} = - \frac{\partial v}{\partial x} = 0,\quad
    \frac{\partial v}{\partial y} = - \frac{\partial v}{\partial y} = 0,\quad
\]

\[
    \therefore u = c_1,\quad v = c_2
\]

\newpage
\section{11.2.7}

\[
    \frac{\partial u}{\partial x} = \frac{\partial v}{\partial y},
    \quad \frac{\partial u}{\partial y} = -\frac{\partial v}{\partial x}
\]

\[
    u = R\cos{\Theta},\quad v = R\sin{\Theta}
\]

\[
    \frac{\partial u}{\partial R} = \cos{\Theta},\quad
    \frac{\partial u}{\partial \Theta} = -R\sin{\Theta},\quad
    \frac{\partial v}{\partial R} = \sin{\Theta},\quad
    \frac{\partial v}{\partial \Theta} = R\cos{\Theta}
\]

\[
    \frac{\partial u}{\partial x} = \frac{\partial u}{\partial R}\frac{\partial R}{\partial x} + \frac{\partial u}{\partial \Theta}\frac{\partial \Theta}{\partial x}
\]
\[
    \frac{\partial u}{\partial x} = \cos{\Theta}\frac{\partial R}{\partial x} - R\sin{\Theta}\frac{\partial \Theta}{\partial x}
\]

\[
    \frac{\partial u}{\partial y} = \frac{\partial u}{\partial R}\frac{\partial R}{\partial y} + \frac{\partial u}{\partial \Theta}\frac{\partial \Theta}{\partial y}
\]
\[
    \frac{\partial u}{\partial y} = \cos{\Theta}\frac{\partial R}{\partial y} - R\sin{\Theta}\frac{\partial \Theta}{\partial y}
\]

\[
    \frac{\partial v}{\partial x} = \frac{\partial v}{\partial R}\frac{\partial R}{\partial x} + \frac{\partial v}{\partial \Theta}\frac{\partial \Theta}{\partial x}
\]
\[
    \frac{\partial v}{\partial x} = \sin{\Theta}\frac{\partial R}{\partial x} + R\cos{\Theta}\frac{\partial \Theta}{\partial x}
\]

\[
    \frac{\partial v}{\partial y} = \frac{\partial v}{\partial R}\frac{\partial R}{\partial y} + \frac{\partial v}{\partial \Theta}\frac{\partial \Theta}{\partial y}
\]
\[
    \frac{\partial v}{\partial y} = \sin{\Theta}\frac{\partial R}{\partial y} + R\cos{\Theta}\frac{\partial \Theta}{\partial y}
\]

From first relation,

\[
    \cos{\Theta}\frac{\partial R}{\partial x} - R\sin{\Theta}\frac{\partial \Theta}{\partial x} = \sin{\Theta}\frac{\partial R}{\partial y} + R\cos{\Theta}\frac{\partial \Theta}{\partial y}
\]

\[
    \cos{\Theta}(\frac{\partial R}{\partial r}\frac{\partial r}{\partial x} + \frac{\partial R}{\partial \theta}\frac{\partial \theta}{\partial x})
    - R\sin{\Theta}(\frac{\partial \Theta}{\partial r}\frac{\partial r}{\partial x} + \frac{\partial \Theta}{\partial \theta}\frac{\partial \theta}{\partial x})
\]
\[
    = \sin{\Theta}(\frac{\partial R}{\partial r}\frac{\partial r}{\partial y} + \frac{\partial R}{\partial \theta}\frac{\partial \theta}{\partial y})
    + R\cos{\Theta}(\frac{\partial \Theta}{\partial r}\frac{\partial r}{\partial y} + \frac{\partial \Theta}{\partial \theta}\frac{\partial \theta}{\partial y})
\]

\[
    \cos{\Theta}(\frac{\partial R}{\partial r}\cos(\theta) - \frac{\partial R}{\partial \theta}\frac{\sin \theta}{r})
    - R\sin{\Theta}(\frac{\partial \Theta}{\partial r}\cos(\theta) - \frac{\partial \Theta}{\partial \theta}\frac{\sin \theta}{r})
\]
\[
    = \sin{\Theta}(\frac{\partial R}{\partial r}\sin(\theta) + \frac{\partial R}{\partial \theta}\frac{\cos \theta}{r})
    + R\cos{\Theta}(\frac{\partial \Theta}{\partial r}\sin(\theta) + \frac{\partial \Theta}{\partial \theta}\frac{\cos \theta}{r})
\]

\[
    (\cos(\Theta)\cos(\theta) - \sin(\Theta)\sin(\theta))\frac{\partial R}{\partial r}
    - \frac{1}{r}(\sin(\Theta)\cos(\theta) + \sin(\theta)\cos(\Theta))\frac{\partial R}{\partial \theta}
\]
\[
    = \frac{R}{r}(\cos(\Theta)\cos(\theta) - \sin(\Theta)\sin(\theta))\frac{\partial \Theta}{\partial \theta}
    + R(\sin(\Theta)\cos(\theta) + \sin(\theta)\cos(\Theta))\frac{\partial \Theta}{\partial r}
\]

\[
    (\cos(\Theta)\cos(\theta) - \sin(\Theta)\sin(\theta))
    (\frac{\partial R}{\partial r} - \frac{R}{r}\frac{\partial \Theta}{\partial \theta})
    - (\sin(\Theta)\cos(\theta) + \sin(\theta)\cos(\Theta))(\frac{1}{r}\frac{\partial R}{\partial \theta} + R\frac{\partial \Theta}{\partial r})
    = 0 \tag{5}\label{eq.5}
\]

From second relation,

\[
    \cos \Theta (\frac{\partial R}{\partial r}\frac{\partial r}{\partial y} + \frac{\partial R}{\partial \theta}\frac{\partial \theta}{\partial y})
    - R \sin \Theta (\frac{\partial \Theta}{\partial r}\frac{\partial r}{\partial y} + \frac{\partial \Theta}{\partial \theta}\frac{\partial \theta}{\partial y})
\]
\[
    = -\sin \Theta (\frac{\partial R}{\partial r}\frac{\partial r}{\partial x} + \frac{\partial R}{\partial \theta}\frac{\partial \theta}{\partial x})
    - R \cos \Theta (\frac{\partial \Theta}{\partial r}\frac{\partial r}{\partial x} + \frac{\partial \Theta}{\partial \theta}\frac{\partial \theta}{\partial x})
\]

\[
    \cos \Theta (\frac{\partial R}{\partial r}\sin(\theta) + \frac{\partial R}{\partial \theta}\frac{\cos \theta}{r})
    - R \sin \Theta (\frac{\partial \Theta}{\partial r}\sin(\theta) + \frac{\partial \Theta}{\partial \theta}\frac{\cos \theta}{r})
\]
\[
    = -\sin \Theta (\frac{\partial R}{\partial r}\cos(\theta) - \frac{\partial R}{\partial \theta}\frac{\sin \theta}{r})
    - R \cos \Theta (\frac{\partial \Theta}{\partial r}\cos(\theta) - \frac{\partial \Theta}{\partial \theta}\frac{\sin \theta}{r})
\]

\[
    (\sin(\theta)\cos(\Theta) + \sin(\Theta)\cos(\theta))
    (\frac{\partial R}{\partial r} - \frac{R}{r}\frac{\partial \Theta}{\partial \theta})
    + (\cos(\Theta)\cos(\theta) - \sin(\theta)\sin(\Theta))(\frac{1}{r}\frac{\partial R}{\partial \theta} + R\frac{\partial \Theta}{\partial r})
    = 0 \tag{6}\label{eq.6}
\]

\begin{enumerate}
    \item
          \[
              (\cos(\Theta)\cos(\theta) - \sin(\theta)\sin(\Theta))\text{eq.}\ref{eq.5}+(\sin(\Theta)\cos(\theta) + \sin(\theta)\cos(\Theta))\text{eq.}\ref{eq.6}
          \]

          \[
              ({(\sin(\theta)\cos(\Theta) + \sin(\Theta)\cos(\theta))}^2 + {(\cos(\Theta)\cos(\theta) - \sin(\Theta)\sin(\theta))}^2)
              (\frac{\partial R}{\partial r} - \frac{R}{r}\frac{\partial \Theta}{\partial \theta})
              = 0
          \]

          Since,

          \[
              \sin(\theta)\cos(\Theta) + \sin(\Theta)\cos(\theta) = \sin(\theta + \Theta)
          \]

          \[
              \cos(\Theta)\cos(\theta) - \sin(\Theta)\sin(\theta) = \cos(\theta + \Theta)
          \]

          \[
              {(\sin(\theta)\cos(\Theta) + \sin(\Theta)\cos(\theta))}^2 + {(\cos(\Theta)\cos(\theta) - \sin(\Theta)\sin(\theta))}^2 = 1
          \]

          \[
              \frac{\partial R}{\partial r} = \frac{R}{r}\frac{\partial \Theta}{\partial \theta}\tag{7}\label{eq.7}
          \]
    \item
          Substituting angle addition identity and eq.\ref{eq.7} in eq.\ref{eq.6} we get,

          \[
              -\cos(\Theta + \theta)(\frac{1}{r}\frac{\partial R}{\partial \theta} + R\frac{\partial \Theta}{\partial r}) = 0
          \]

          Since this equation holds for all $\theta$ and $\theta$ then,

          \[
              \frac{1}{r}\frac{\partial R}{\partial \theta} \mbeq - R\frac{\partial \Theta}{\partial r}
          \]

\end{enumerate}

\newpage
\section{11.2.8}

\[
    \frac{\partial R}{\partial r} = \frac{R}{r}\frac{\partial \Theta}{\partial \theta}, \quad
    \frac{1}{r}\frac{\partial R}{\partial \theta} = - R\frac{\partial \Theta}{\partial r}
\]

\[
    \frac{\partial \Theta}{\partial \theta} = \frac{r}{R}\frac{\partial R}{\partial r}
\]

\begin{equation}
    \begin{split}
        \frac{1}{r^2}\frac{\partial^2 \Theta}{\partial \theta^2}
         & = \frac{1}{r} \frac{\partial }{\partial \theta}(\frac{1}{R}\frac{\partial R}{\partial r})                                                    \\
         & = \frac{1}{Rr}(\frac{\partial^2 R}{\partial \theta \partial r} - \frac{1}{R}\frac{\partial R}{\partial \theta}\frac{\partial R}{\partial r}) \\
    \end{split}
\end{equation}

\begin{equation}
    \begin{split}
        \frac{\partial^2 \Theta}{\partial r^2}
         & =\frac{\partial }{\partial r}(\frac{\partial \Theta}{\partial r})                                                                                              \\
         & = -\frac{\partial }{\partial r}(\frac{1}{Rr}\frac{\partial R}{\partial \theta})                                                                                \\
         & = -\frac{\partial }{\partial r}(\frac{1}{Rr})\frac{\partial R}{\partial \theta} - \frac{1}{Rr}\frac{\partial }{\partial r}(\frac{\partial R}{\partial \theta}) \\
         & = \frac{1}{Rr}(\frac{1}{R}\frac{\partial R}{\partial r} \frac{\partial R}{\partial \theta}
        + \frac{1}{r}\frac{\partial R}{\partial \theta}
        - \frac{\partial^2 R}{\partial r \partial \theta})
    \end{split}
\end{equation}

\[
    \frac{1}{r}\frac{\partial \Theta}{\partial r} = -\frac{1}{Rr^2}\frac{\partial R}{\partial \theta}
    \tag{3}
\]


\[
    \frac{\partial^2 \Theta}{\partial r^2} + \frac{1}{r}\frac{\partial \Theta}{\partial r} + \frac{1}{r^2}\frac{\partial^2 \Theta}{\partial \theta^2} = 0
\]


\newpage
\section{11.2.11}

\begin{enumerate}
    \item

          \[
              \frac{df}{dz} = \frac{\partial f}{\partial x} = \frac{\partial u}{\partial x} + i \frac{\partial v}{\partial x} = V_x - i \frac{\partial u}{\partial y} = V_x - i V_y
          \]

    \item

          Since the Laplacian of an analytic function of z is zero

          \[
              \nabla . V =  \nabla . \nabla u = \nabla^2 u = 0
          \]

    \item

          From the curl of gradient identity

          \[
              \nabla \times \nabla u = 0
          \]

\end{enumerate}

\newpage
\section{11.3.6}

\[
    z^* = x-iy
\]

For path 1, \\
Segment 1: \(dz = dx, \quad x_1 = 0, \quad x_2 = 1, \quad y = 0\) \\
Segment 2: \(dz = idy, \quad y_1 = 0, \quad y_2 = 1, \quad x = 1\)

\[
    \int_0^{1+i} z^* \, dz= \int_0^{1} xdx + i\int_0^{1} (1-iy)dy
    = 1 + i
\]

For path 2, \\
Segment 1: \(dz = idy, \quad y_1 = 0, \quad y_2 = 1, \quad x = 0\) \\
Segment 2: \(dz = dx, \quad x_1 = 0, \quad x_2 = 1, \quad y = 1\)

\[
    \int_0^{1+i} z^* \, dz = \int_0^{1} y dy + \int_0^{1} (x - i)dx
    = 1 - i
\]


\newpage
\section{11.3.7}

\[
    \int_C \frac{1}{z(z+1)}dz
\]

\[
    \int_C \frac{1}{z} dz - \int_C \frac{1}{z+1} dz = 2 \pi i - 2 \pi i = 0
\]

\newpage
\section{11.4.1}

\[
    \frac{1}{2\pi i}\oint z^{m-n-1} \, dz
\]

for m = n,

\[
    \frac{1}{2\pi i}\oint z^{-1} \, dz = \frac{1}{2\pi i} 2\pi i = 1
\]

for $m \neq n$ the exponent won't be $-1$ so,

\[
    \oint z^{m-n-1} \, dz = 0
\]

\[
    \frac{1}{2\pi i}\oint z^{m-n-1} \, dz = \delta_{mn}
\]

\newpage
\section{11.4.6}

The contours encapsulate the singularity point of $z=0$

\[
    f(z) = e^{iz}
\]

\[
    n = 3
\]

\[
    \oint \frac{f(z)}{z^n} \, dz = \frac{2 \pi i}{(n-1)!}f^{(n-1)}(0)
\]

\[
    \oint \frac{e^{iz}}{z^3} \, dz = -\pi i
\]

\newpage

\bibliographystyle{plain}
\bibliography{references}
\nocite{arfken2013mathematical}
\nocite{El-Deeb_PEU-455_Assignments}

\end{document}