\documentclass[12pt]{article}
\usepackage[svgnames,x11names,table]{xcolor}
\usepackage{hyperref}
\usepackage{graphicx}
\usepackage{parskip}
\usepackage{float}
\usepackage{amsmath}
\usepackage{amssymb}
\usepackage{enumitem}
\usepackage[thicklines]{cancel}

\hypersetup{
    colorlinks,
    citecolor=black,
    filecolor=black,
    linkcolor=RoyalBlue4,
    urlcolor=RoyalBlue4,
}

\title{PEU 455 Assignment 1}
\author{Mohamed Hussien El-Deeb (201900052)}
\date{\today}

\begin{document}

\maketitle
\tableofcontents

\newcommand{\Lagr}{\mathcal{L}}

\section{8.2.3}

Chebyshev:
\[(1-x^2)y''-xy'+n^2y=0\]

Operator:
\[\Lagr(x)=P_0(x)\frac{d^2}{dx^2}+P_1(x)\frac{d}{dx}+P_2(x)\]

Self-adjoint Condition:
\[P_0'(x) = P_1(x)\]

Weighting Function:
\[w(x)={(1-x^2)}^{-\frac{1}{2}}\]

$\text{Chebyshev}*w(x)$:
\[
    {(1-x^2)}^{\frac{1}{2}}y''-x{(1-x^2)}^{-\frac{1}{2}}y'+n^2{(1-x^2)}^{-\frac{1}{2}}y=0
\]

\[
    P_0(x)={(1-x^2)}^{\frac{1}{2}}
\]

\[
    P_0'(x)= -x{(1-x^2)}^{-\frac{1}{2}}
\]

\[
    P_1(x)= -x{(1-x^2)}^{-\frac{1}{2}}
\]

We can since that Self-adjoint Condition holds.

\newpage
\section{8.2.5}

Given:

\[
    \Lagr u_1(x) = \lambda_1 u_1(x)
\]

\[
    \Lagr u_2(x) = \lambda_2 u_2(x)
\]

\[
    \lambda_1 \neq \lambda_2 \tag{1}
\]

To prove:

\[
    u_1(x) \neq \alpha u_2(x)
\]

Prove by contradiction:

\[
    Let\quad u_1(x) = \alpha u_2(x) \tag{2}
\]

\[
    \therefore \Lagr u_1(x) = \alpha \Lagr u_2(x) = \alpha \lambda_2 u_2(x) = \lambda_1 u_1(x)
\]

\[
    \therefore u_1(x) = \frac{\lambda_2}{\lambda_1} \alpha u_2(x)
\]

In order to satisfy (2) $\lambda_1 \overset{!}{=} \lambda_2$ which contradicts (1).

\newpage
\section{8.2.7}

\[
    \int^1_{-1} T^*_0(x) V_1(x) w(x)\, dx
\]

\[
    \int^1_{-1} {(1-x^2)}^{\frac{1}{2}} {(1-x^2)}^{-\frac{1}{2}}\, dx = \int^1_{-1} dx = 2
\]

Since the result of the integral is not zero, therefore $T_0$ and $V_1$ are not orthogonal on the range $(-1, 1)$ with the weighting function ${(1-x^2)}^{-\frac{1}{2}}$.

\newpage
\section{8.2.8}

Given:

\[
    \frac{d}{dx}\left[P(x)\frac{d}{dx}u_n(x)\right]+\lambda_n w(x)u_n(x)=0
\]

\[
    \Lagr u_m(x) = \lambda_m u_m(x)
\]

\[
    \Lagr u_n(x) = \lambda_n u_n(x)
\]

\[
    \lambda_1 \neq \lambda_2
\]

\[
    \int_a^b w(x)u_m^*(x) u_n(x)\, dx = 0
\]

To Prove:

\[
    \int_{a'}^{b'} {u_m'}^*(x) u_n'(x) P(x)\, dx = 0
\]

Solution:

\[
    \frac{d}{dx}\left[P(x) u_n'(x)\right]+\lambda_n w(x)u_n(x)=0
\]


\[
    \int_{a}^{b} u_m^*(x)\frac{d}{dx}\left[P(x) u_n'(x)\right]\, dx + \lambda_n \int_{a}^{b} w(x)u_m^*(x)u_n(x)\, dx = 0
\]

\[
    \int_{a}^{b} u_m^*(x)\frac{d}{dx}\left[P(x) u_n'(x)\right]\, dx = 0
\]

\[
    \int_{a}^{b}P(x) u_n'(x)\frac{d}{dx}\left[ u_m^*(x)\right]\, dx = P(x) u_n'(x)u_m^*(x)
\]

\[
    \int_{a}^{b}P(x) u_n'(x) {u_m'}^*(x) \, dx = {\left[P(x) u_n'(x)u_m^*(x)\right]}_a^b
\]

Under this boundary condition:

\[
    {\left[P(x) u_n'(x)u_m^*(x)\right]}_a^b = 0
\]

Dot product of $u_n'(x)$ and $u_m'$ lead to orthogonality under the weighting function $P(x)$

\newpage
\section{8.3.4}

Given:

\[x L^{''}_n(x) + (1-x) L^{'}_n(x) + n L_n(x) = 0\]

Let:

\[L_n(x)=\sum_{j = 0}^{\infty} a_j^{(n)} x^{s+j} \]

Therefore:

\[L^{'}_n(x)=\sum_{j = 0}^{\infty} (s+j)a_j^{(n)} x^{s+j-1} \]
\[L^{''}_n(x)=\sum_{j = 0}^{\infty} (s+j)(s+j-1)a_j^{(n)} x^{s+j-2} \]

\[x \sum_{j = 0}^{\infty} (s+j)(s+j-1)a_j^{(n)} x^{s+j-2} + (1-x) \sum_{j = 0}^{\infty} (s+j)a_j^{(n)} x^{s+j-1} + n \sum_{j = 0}^{\infty} a_j^{(n)} x^{s+j} = 0\]

\[\sum_{j = 0}^{\infty} (s+j)(s+j-1)a_j^{(n)} x^{s+j-1} + \sum_{j = 0}^{\infty} (s+j)a_j^{(n)} x^{s+j-1} - \sum_{j = 0}^{\infty} (s+j)a_j^{(n)} x^{s+j} + \sum_{j = 0}^{\infty} n a_j^{(n)} x^{s+j} = 0\]

\[\sum_{j = 0}^{\infty} a_j^{(n)} {(s+j)}^2 x^{s+j-1} - \sum_{j = 0}^{\infty} a_j^{(n)} (s+j-n) x^{s+j} = 0\]

\[\sum_{j = -1}^{\infty} a_{j+1}^{(n)} {(s+j+1)}^2 x^{s+j} - \sum_{j = 0}^{\infty} a_j^{(n)} (s+j-n) x^{s+j} = 0\]

\[a_{0}^{(n)} s^2 x^{s-1} + \sum_{j = 0}^{\infty} a_{j+1}^{(n)} {(s+j+1)}^2 x^{s+j} - \sum_{j = 0}^{\infty} a_j^{(n)} (s+j-n) x^{s+j} = 0\]

\[a_{0}^{(n)} s^2 x^{s-1} +  x^{s} \sum_{j = 0}^{\infty} (a_{j+1}^{(n)} {(s+j+1)}^2 - a_j^{(n)} (s+j-n)) x^{j} = 0\]
\[a_{0}^{(n)} s^2 + \sum_{j = 0}^{\infty} \left(a_{j+1}^{(n)} {(s+j+1)}^2 - a_j^{(n)} (s+j-n)\right) x^{j + 1} = 0\]

We need to make the summation be always zero regardless of x so the constant term must be zero.

Therefore $a_{0}^{(n)}$ or $s^2$ must be zero.

\[\sum_{j = 0}^{\infty} \left(a_{j+1}^{(n)} {(s+j+1)}^2 - a_j^{(n)} (s+j-n)\right) x^{j + 1} = 0\]

For the same reason

\[a_{j+1}^{(n)} {(s+j+1)}^2 - a_j^{(n)} (s+j-n) = 0\]

\[a_{j+1}^{(n)} = \frac{s+j-n}{{(s+j+1)}^2}a_j^{(n)}\]

From this we can't set $a_0 = 0$ because all subsequent $a_j$ must be zero as well making the solution a trivial one.

Therefore s must be zero.

\[a_{j+1}^{(n)} = \frac{j-n}{{(j+1)}^2}a_j^{(n)}\]

Where j starts from 0.

For the series to be finite the parameter n must be a positive integer.

and the polynomial solution will be of order n.

\newpage
\section{8.3.6}

Given:

\[(1-x^2) U^{''}_n(x) - 3x U^{'}_n(x) + n(n+2) U_n(x) = 0\]

Let:

\[U_n(x)=\sum_{j = 0}^{\infty} {a_j^{(n)} x^{s+j}} \]

Therefore:

\[U^{'}_n(x)=\sum_{j = 0}^{\infty} (s+j)a_j^{(n)} x^{s+j-1} \]
\[U^{''}_n(x)=\sum_{j = 0}^{\infty} (s+j)(s+j-1)a_j^{(n)} x^{s+j-2} \]

\[(1-x^2) \sum_{j = 0}^{\infty} (s+j)(s+j-1)a_j^{(n)} x^{s+j-2} - 3x \sum_{j = 0}^{\infty} (s+j)a_j^{(n)} x^{s+j-1} + n(n+2) \sum_{j = 0}^{\infty} a_j^{(n)} x^{s+j} = 0\]

\begin{multline*}
    \sum_{j = 0}^{\infty} (s+j)(s+j-1)a_j^{(n)} x^{s+j-2} - \sum_{j = 0}^{\infty} (s+j)(s+j-1)a_j^{(n)} x^{s+j} \\
    - \sum_{j = 0}^{\infty} 3(s+j)a_j^{(n)} x^{s+j} + \sum_{j = 0}^{\infty} n(n+2) a_j^{(n)} x^{s+j} = 0
\end{multline*}

\[
    \sum_{j = 0}^{\infty} (s+j)(s+j-1)a_j^{(n)} x^{s+j-2}
    + \sum_{j = 0}^{\infty} \left(n(n+2)  - (s+j)(s+j+2)\right)a_j^{(n)} x^{s+j} = 0
\]

\[
    \sum_{j = -2}^{\infty} (s+j+2)(s+j+1)a_{j + 2}^{(n)} x^{s+j}
    + \sum_{j = 0}^{\infty} \left(n(n+2)  - (s+j)(s+j+2)\right)a_j^{(n)} x^{s+j} = 0
\]

\begin{multline*}
    s(s-1) a_0^{(n)} x^{s-2} + s(s+1) a_1^{(n)} x^{s-1}
    + \sum_{j = 0}^{\infty} (s+j+2)(s+j+1)a_{j+2}^{(n)} x^{s+j} \\
    + \sum_{j = 0}^{\infty} \left(n(n+2)  - (s+j)(s+j+2)\right)a_j^{(n)} x^{s+j} = 0
\end{multline*}

\begin{multline*}
    s(s-1) a_0^{(n)} x^{s-2} + s(s+1) a_1^{(n)} x^{s-1} + \\
    \sum_{j = 0}^{\infty} \left((s+j+2)(s+j+1)a_{j+2}^{(n)}
    + \left(n(n+2)  - (s+j)(s+j+2)\right)a_j^{(n)}\right) x^{s+j} = 0
\end{multline*}

Similar to last question we figure that:

\[
    s(s-1) a_0^{(n)} = 0
\]

\[
    s(s+1) a_1^{(n)} = 0
\]

\[
    a_{j+2}^{(n)} = \frac{(s+j+2)(s+j) - n(n+2)}{(s+j+1)(s+j+2)}a_j^{(n)}
\]

% \[
%     a_{j+2}^{(n)} = \frac{j(j+2) - n(n+2)}{(j+1)(j+2)}a_j^{(n)}
% \]

% n = j

% n is even

% \[
%     a_{j+2}^{(n)} = \frac{(j+3)(j + 1) - n(n+2)}{(j+2)(j+3)}a_j^{(n)}
% \]

% n = j + 1

% n is odd

For the polynomial to be finite:

\[
    (s+j+2)(s+j) = n(n+2)
\]

So, the truncation value of j is

\[
    j = n -s
\]

Therefore $n-s$ is positive odd integer for odd solutions

Choosing odd solutions means that we could set $a_0 = 0$ so s could be 0 or -1

We will choose s = 0

\[
    a_{j+2}^{(n)} = \frac{j(j+2) - n(n+2)}{(j+1)(j+2)}a_j^{(n)}
\]

Where n is an odd positive number and j starts from 1

\newpage

\bibliographystyle{plain}
\bibliography{references}
\nocite{arfken2013mathematical}
\nocite{El-Deeb_PEU-455_Assignments}

\end{document}